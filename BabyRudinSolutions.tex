\documentclass[a4paper,11pt]{article}
\author{Adam Cimpeanu}
\title{Principles of Mathematical Analysis Solutions}
\usepackage{amsmath}
\usepackage{amsthm}
\usepackage{amsfonts}
\usepackage{IEEEtrantools}
\usepackage{hyperref}

\hypersetup{colorlinks, linkcolor=blue}
\begin{document}
\maketitle
\tableofcontents

\newpage

\section{Chapter 1: The Real and Complex Number Systems}


\subsection{Problem 1}
Given $r \neq 0$ that is rational and can be expressed as a reduced fraction
$p/q$ and irrational x.

Assume $r+x$ is rational, and can therefore be expressed as $n/m$ where at most
one of n and m are even.
\begin{IEEEeqnarray}{rCl}
    n/m & = & r+x = p/q + x \\
    x & = & n/m - r/x = \frac{n \cdot x - r \cdot m}{m \cdot x}
\end{IEEEeqnarray}
Thus x can be expressed as a ratio of integers and is thus rational, a
    contradiction.

The multiplicative case is similar. Assume r*x is rational. Keeping the same
    notation
\begin{IEEEeqnarray}{rCl}
    r \cdot x & = & n/m \\
    p/q \cdot x & = & n/m \\
    x & = & \frac{n \cdot q}{p \cdot m}
\end{IEEEeqnarray}
Showing that x is also a ratio of integers, a contradiction.


\subsection{Problem 2}
We start by proving that $\sqrt{3}$ is irrational.

Suppose it is rational, such that $\sqrt{3}$ can be expressed as a fully
reduced $n/m$. This gives us:
\begin{equation}
    3 m^2 = n^2
\end{equation}
$m$ and $n$ must have the same parity, and they cannot both be even without
contradicting the assumption that they're both fully reduced. Therefore, they
are both odd and can be reexpressed as follows:
\begin{IEEEeqnarray}{rCl}
    m & = & 2 k_{1} + 1 \\
    n & = & 2 k_{2} + 1
\end{IEEEeqnarray}
Combining these equations and simplifying yields:
\begin{IEEEeqnarray}{rCl}
    3 (2 k_{1} + 1)^2 & = & (2 k_{2} + 1)^2 \\
    12 k_{1}^2 + 12 k_{1} + 3 & = & 4 k_{2}^2 + 4 k_{2} + 1 \\
    6 k_{1}^2 + 6 k_{1} + 1 & = & 2 k_{2}^2 + 2 k_{2}
\end{IEEEeqnarray}
The left side is odd and the right side is even, a contradiction.

Equipped with this knowledge, we turn back to $\sqrt{12}$. Assume it is rational
and can be expressed as a ratio of reduced $n$ and $m$. As before, we write
\begin{equation}
    12 m^2 = n^2
\end{equation}
$n$ must be even, and so if we subsitute $n=2k$
\begin{IEEEeqnarray}{rCl}
    12 m^2 & = & 4 k^2 \\
    3 & = & \frac{k^2}{m^2}
\end{IEEEeqnarray}
Implying that $\sqrt{3}$ is rational, a contradiction.


\subsection{Problem 3}
\begin{IEEEeqnarray}{rCl}
    y & = & 1 \cdot y = ((1/x) \cdot x) \cdot y = (1/x) \cdot (xy) \\
    & = & (1/x) \cdot (xz) = ((1/x) \cdot x) \cdot z = 1 \cdot z = z
\end{IEEEeqnarray}
Which proves (a). Setting $z=1$ gives us (b). (c) is a result of multiplying
both sides of $xy = 1$ by $1/x$. (d) is derived from (c) if we substitute
$x$ for $1/y$.


\subsection{Problem 4}
Given that $E$ is a nonempty subset of an ordered set, and that $\alpha$ is
the lower bound of $E$ and $\beta$ is the upper bound. By the definition of
lower bound and upper bounds, we know that for any $x \in E$, $\alpha \leq x$
and $x \leq \beta$. Thus $\alpha \leq \beta$.


\subsection{Problem 5}
Given nonempty $A$ bounded below, let us define $\alpha = \inf A$. $\alpha$ has
the property that for any lower bound $\gamma$, $\gamma \leq \alpha$. $-\alpha$
must be an upper bound on $-A$, because $\alpha \leq x \ \forall x \in A$ implies
$x \leq -\alpha \ \forall x \in -A$.

And as $\alpha = \inf A$, no $\beta > \alpha$ is a lower bound for $A$, which
implies that no $-\beta < -\alpha$ is an upper bound for $-A$, which means,
$-\alpha = \sup -A$.


\subsection{Problem 6}
For this problem, we need to define negative and zero exponents. If we hold
$b^{x}b^{y}=b^{x+y}$ for integers $x$, $y$ and $b>1$, then $b^{1}b^{0}$
should equal $b^{1}$, which implies $b^{0}=1$ from 1.15(b). In a similar
vein, we would expect $b^{x}b^{-x}=b^{0}$, which combined with 1.15(c),
gives us that $b^{-x}=\frac{1}{b}^{x}$.

For $b > 1$ and $n$,$m$ integers, we first establish a few results.
\begin{IEEEeqnarray}{rCl}
    (b^m)^{1/n} & = & (\underbrace{b \cdot b \cdot \ldots \cdot b} _\text{m})^{1/n}
    = \underbrace{b^{1/n} \cdot b^{1/n} \cdot \ldots b^{1/n}} _\text{m} = (b^{1/n})^{m}
    \label{eq:1.6.1} \\
    y^{n} & = & a = b \Rightarrow y = a^{1/n} = b^{1/n} \label{eq:1.6.2} \\
    y^{m} & = & b^{n \cdot m} = (b^{n})^{m} \Rightarrow y = b^{n} \label{eq:1.6.3}
\end{IEEEeqnarray}
The latter two are corallaries of Theorem 1.21, which proves the uniqueness of
the nth root.

From the problem, we know that $m$, $n$, $p$, and $q$ are integers with $n > 0$
and $q > 0$, and that $r = m/n = p/q$. If either $m$ or $p$ are $0$, the
result is trivially true. If $m$ is negative, then $p$ must be as well, and by
the definition established above, we could rewrite the proof for $c=1/b$ and
positive $m$ and $p$. Thus, we only treat the case where $m$ and $p$ are positive.

We know that $mq = pn$. Using the $b$ from before, we can write
$b^{m q} = b^{p n}$. Using results in (\ref{eq:1.6.2}) and (\ref{eq:1.6.3}), we can
take the qth root of both sides, giving us $b^{m} = (b^{p n})^{1/q}$. Using
(\ref{eq:1.6.1}), we rewrite as $b^{m} = (b^{p/q})^{n}$ and take the nth root by
the same principles, yielding $b^{m/n} = b^{p/q}$ and proving (a).

Let $r$ and $s$ be rational and represented by fully reduced fractions $m/n$
and $p/q$ respectively, where $n$, $m$, $p$, and $q$ are all integers, and
$m > 0$, $q > 0$ as before. We take advantage of these facts when exponents are
integers:

\begin{IEEEeqnarray}{rCl}
    b^{n} b^{p} & = & \underbrace{b \cdot b \cdot \ldots \cdot b} _\text{n}
    \cdot \underbrace{b \cdot b \cdot \ldots \cdot b} _\text{p} =
    \cdot \underbrace{b \cdot b \cdot \ldots \cdot b} _\text{n+p} = b^{n+p} \\
    (b^{n})^{p} & = & \underbrace{b^{n} \cdot b^{n} \cdot \ldots \cdot b^{n}} _\text{p}
    = \underbrace{b \cdot b \cdot \ldots \cdot b} _\text{pn} = b^{pn} \\
    b^{\frac{1}{np}} & = & (b^{1/n})^{1/p}
\end{IEEEeqnarray}
The last of which is derived from iterations of Theorem 1.21. Then we have:
\begin{IEEEeqnarray}{rCl}
    b^{r}b^{s} & = & b^{m/n}b^{p/q} = (b^{m/n})^{q/q}(b^{p/q})^{n/n}
    = (b^{mq})^{\frac{1}{nq}}(b^{pn})^{\frac{1}{nq}} \IEEEnonumber\\
    & = & b^{\frac{mq+pn}{nq}} = b^{m/n + p/q} = b^{r+s}
\end{IEEEeqnarray}
Which proves (b).

For real $x$, we define $B(x)$ to be the set of $b^{t}$, where $t$ is rational
and $t \leq x$. Recall that $b$ is a real number greater than 1. Critical to
this part is the fact that for rational $n$ and $m$:
\begin{equation}
    b^{n}<b^{m} \iff n<m \label{eq:1.6.4}
\end{equation}
Combining the fact that for $S(t) = \{a \in \mathbb{Q} \ | \ a \leq t\}$,
$\sup S(t)= t$, with (\ref{eq:1.6.4}) yields the result that $\sup B(r) = b^{r}$
for rational $r$. Given that $\mathbb{R}$ has the least upper bound property,
we know that $\sup S(x)$ for real $x$ exists and is $x$. By (\ref{eq:1.6.4}),
we then conclude that $\sup B(x) = b^{x}$, proving (c).

For the final part, let $x,y \in \mathbb{R}$:
\begin{IEEEeqnarray}{rCl}
    b^{x+y} & = & \sup B(x+y) = \sup \{b^{t} | t \in \mathbb{Q}, t \leq x+y\} \\
    & = & \sup \{b^{t+s} | t,s \in \mathbb{Q}, t+s \leq x+y\} \\
    & = & \sup \{b^{t}b^{s} | t,s \in \mathbb{Q}, t+s \leq x+y\} \\
    & = & \sup \{b^{t} | t \in \mathbb{Q}, t \leq x\}
    \sup \{b^{s} | s \in \mathbb{Q}, s \leq y\} \\
    & = & \sup B(x) \sup B(y) = b^{x}b^{y}
\end{IEEEeqnarray}


\subsection{Problem 7}
Let $b>1$ and $n$ be a positive integer n. $b^{n}-1 \geq n(b-1)$ if $n=1$.
By way of induction, we assume the inequality is true for $n$ and prove
$n+1$.
\begin{IEEEeqnarray}{rCl}
    b^{n+1}-1 > b^{n+1}-b = b(b^{n}-1) \geq bn(b-1) > (n+1)(b-1)
\end{IEEEeqnarray}
The last part is true because $n(b-1)-1>0$. So we have (a)

Substitute $c=b^{1/n}$ to get (b).

If $t>1$ and $n > \frac{b-1}{t-1}$, then by (b),
$n(t-1) > n(b^{1/n}-1)$, which simplifies to $b^{1/n}<t$, proving (c).

(d) follows as described by the problem.

Set $t = y^{-1}b^{w}$ and repeat the logic in (d) to get (e).

For $A=\{w|b^{w}<y\}$, define $x=\sup A$. We know that $b^{x}$ cannot be larger
than $y$, because then by (d), there would be a $b^{x-1/n}$ that would serve
as a smaller upper bound. So it must be the case that $b^{x} \leq y$. But it
also cannot be the case that $b^{x} < y$, because then there would be, by (c),
a $b^{x+1/n} <y$ that would would invalidate $x$ as an upper bound. So it must
be the case that $\sup A = x$, proving (f).

$x$ must be unique because otherwise, one of $x_1$ or $x_2$ would be smaller, and
thus, one of $b^{x_1}$ and $b^{x_2}$ would be less than $y$.


\subsection{Problem 8}
By our defined ordering, suppose $i>0$. then by 1.17(ii), $i^{2}=-1>0$.
Reapplying the same rule, $-1 \cdot i = -i$, but also be equal to zero, a
contradiction with 1.18(a). The same can be done starting with $i<0$ and
writing it as $-i>0$.


\subsection{Problem 9}
Let $z = a + bi$ and $w = c + di$. If $a<c$, by the lexicographic order
definition, $z<w$, and $z$ cannot be greater than or equal to $w$. If
$a>c$, $z>w$ and $z$ cannot be less than or equal to $w$. Then consider
the cases in which $a=c$. Either it's the case that $b=d$, at which point,
$z=w$, and $z$ cannot be greater than or less than $w$. Or $b<d$, in which
case $z<w$ and $z$ cannot be greater than or equal to $w$. Finally, if $b>d$,
it cannot be the case that $z$ is equal to or less than $w$, it must be greater.
This proves the first condition of an ordered set.

Now we assume that $x < y$ and $y < z$. Either $x$ and $z$ have the same first
terms and $z$'s second term is larger than $x$'s. Or $z$'s first term must be
larger than $x$'s first term, meaning the transitive property holds, making the
lexicographic order define an ordered set.

Define $E \subset \mathbb{C}$ where all $a+bi$ have $a$ equal to some real $r$.
Now imagine the complex number $(r+1)+ci$ for any real $c$. This new number
bounds $E$, but there is no least upper bound on $E$, becaus for any imagined
bound $(r+1)+ci$, $(r+1)+(c+1)i$ is a larger bound.


\subsection{Problem 10}
Let $z = a + bi$ and $w = u + vi$ and $a = \sqrt{\frac{|w|+u}{2}}$
$b = \sqrt{\frac{|w|-u}{2}}$. Then:
\begin{IEEEeqnarray}{rCl}
    z^{2} & = & a^{2} - b^{2} + 2abi = \frac{|w|+u}{2} - \frac{|w|-u}{2} + 
    \sqrt{(|w|+u)(|w|-u)} \ i \\
    & = & u + \sqrt{|w|^{2}-u^{2}}i = u + vi
\end{IEEEeqnarray}
Thus $z^{2}=w$ when $v \geq 0$. Similarly:
\begin{IEEEeqnarray}{rCl}
    \overline{z}^{2} & = & a^{2} - b^{2} - 2abi =
    \frac{|w|+u}{2} - \frac{|w|-u}{2} - \sqrt{(|w|+u)(|w|-u)} \ i \\
    & = & u - \sqrt{|w|^{2}-u^{2}}i = u - vi
\end{IEEEeqnarray}
Therefore $\overline{z}^{2}=w$ when $v \leq 0$. By this construction, every
complex number $w$ has two square roots, either $z$ and $-z$ or $\overline{z}$
and $-\overline{z}$, with the exception of $0$, which only has the root of $0$.


\subsection{Problem 11}
Let $z=a+bi$. Suppose $|z|=\sqrt{a^{2}+b^{2}}=r$. Then let
$w=\frac{a}{r}+\frac{b}{r}i$. Then we get $z = rw$, uniquely constructed.


\subsection{Problem 12}
$|z_1| = |z_1|$ by definition. Now we assume
$|z_1 + z_2 + \ldots + z_n| \leq |z_1| + |z_2| + \ldots + |z_n|$ and aim to
prove the case for $n+1$. By 1.33(e):
\begin{IEEEeqnarray}{rCl}
    |(z_1 + z_2 + \ldots + z_n) + z_{n+1}| & \leq & |z_1 + z_2 + \ldots + z_n| +
        |z_{n+1}| \\
    & \leq & |z_1| + |z_2| + \ldots + |z_n| + |z_{n+1}|
\end{IEEEeqnarray}
Which completes the proof by induction.


\subsection{Problem 13}
Let $x = a + bi$ and $y = c + bi$. We begin with:
\begin{IEEEeqnarray}{rCl}
    (ad - bc)^{2} & \geq & 0 \\
    a^{2}d^{2} - 2abcd + b^{2}c^{2} & \geq & 0 \\
    a^{2}c^{2} + b^{2}c^{2} + a^{2}d^{2} + b^{2}d^{2} & \geq & a^{2}c^{2} +
        2abcd + b^{2}d^{2} = (ac+bd)^{2} \\
    (a^{2}+b^{2})(c^{2}+d^{2}) & \geq & (ac+bd)^{2} \\
    \sqrt{(a^{2}+b^{2})(c^{2}+d^{2})} & \geq & ac + bd
\end{IEEEeqnarray}
From this seemingly odd relation, we can write this seemingly odd relation:
\begin{IEEEeqnarray}{rCl}
    \sqrt{a^{2}+b^{2}+c^{2}+d^{2}-2\sqrt{(a^{2}+b^{2})(c^{2}+d^{2})}} & \leq &
        \sqrt{a^{2}+b^{2}+c^{2}+d^{2}-2(ac+bd)} \IEEEnonumber\\
    |\sqrt{a^{2}+b^{2}} - \sqrt{c^{2}+d^{2}}| & \leq &
        \sqrt{a^{2}-2ac+c^{2}+b^{2}-2bd+d^{2}} \IEEEnonumber\\
    |\sqrt{a^{2}+b^{2}} - \sqrt{c^{2}+d^{2}}| & \leq &
        \sqrt{(a-c)^{2}+(b-d)^{2}} \IEEEnonumber\\
    ||x|-|y|| & \leq & |(a-c)+(b-d)i| = |x-y| \IEEEnonumber
\end{IEEEeqnarray}
Proving the desired inequality


\subsection{Problem 14}
If $z$ is a complex number such that $|z|=1$, then we have:
\begin{IEEEeqnarray}{rCl}
    |1+z|^{2} + |1-z|^{2} & = & (1+z)(1+\overline{z}) + (1-z)(1-\overline{z})\\
    & = & 1 + z + \overline{z} + z\overline{z} + 1 - z - \overline{z} +
        z\overline{z} \\
        & = & 2 + 1 + 1 = 4
\end{IEEEeqnarray}


\subsection{Problem 15}
Using the notation for the proof of theorem 1.35, the Schwarz inequality is
trivially true if either $A$ or $B$ are 0. Otherwise, imagine that the
complex vectors $\mathbf{a}$ and $\mathbf{b}$ are linearly dependent, which is
to say $a_i=k b_i \ \forall i=1 \ldots n$ and some real number $k$. Then we have
\begin{IEEEeqnarray}{rCl}
    \left|\sum_{j=1}^n a_j \overline{b_j}\right|^2 & = &
    \left|k \sum_{j=1}^n a_j \overline{a_j}\right|^2 =
    \left|k \sum_{j=1}^n |a_j|\right|^2 = \\
    k^2 \sum_{j=1}^n |a_j|^2 & = & \sum_{j=1}^n |a_j|^2 \sum_{j=1}^n |k a_j|^2 =
    \sum_{j=1}^n |a_j|^2 \sum_{j=1}^n |b_j|^2
\end{IEEEeqnarray}


\subsection{Problem 16}
An intuition of this problem is best captured in the $k=3$ setting. Imagine
$\mathbf{x}$, $\mathbf{y}$, and $\mathbf{z}$ as three vectors in 3D space.
$|x-y|$ is really just the distance between the points of the vectors. In
this problem, $\mathbf{z}$ must be equidistant from the other two vectors,
which basically means it must line on the plane that lies strictly between
$\mathbf{x}$ and $\mathbf{y}$. If the tip of $\mathbf{z}$ lies exactly at
the midpoint of the line between the tips of $\mathbf{x}$ and $\mathbf{y}$,
which is to say, if it is $\frac{1}{2}(\mathbf{x}+\mathbf{y})$, then it's
distance from each vector $r$ will be $1/2d$. From that image of the triangle,
you can see that there's no $\mathbf{z}$ thats less than $1/2d$ away from
$\mathbf{x}$ and $\mathbf{y}$, and there are infinite options on the splitting
plane for $\mathbf{z}$ that's more than $1/2d$ away from both.

In the $k=1,2$ cases, there are no solutions for (a), and the solutions for
(b) and (c) are the same.


\subsection{Problem 17}
\begin{IEEEeqnarray}{rCl}
    |x+y|^2+|x-y|^2 & = & \sum_{i=1}^n x_i^2 + 2x_i y_i + y_i^2 +
    \sum_{i=1}^n x_i^2 - 2x_i y_i + y_i^2 \\
    & = & 2\sum_{i=1}^n x_i^2 + y_i^2 = 2|x|^2 + 2|y|^2
\end{IEEEeqnarray}
This result can be interpreted as saying that the volume of a parallelogram
spanned by $\mathbf{x}$ and $\mathbf{y}$ is equal to the hypotenuse of the
right triangle created by those two vectors.


\subsection{Problem 18}
If we expand $\mathbf{x} \cdot \mathbf{y}=0$ and move the terms, we get:
\begin{IEEEeqnarray}{rCl}
    x_1 y_1 & = & -\sum_{i=2}^n x_i y_i \\
    y_1 & = & \frac{-\sum_{i=2}^n x_i y_i}{x_1}
\end{IEEEeqnarray}
So for any arbitrary $\mathbf{x}$ and a mostly arbitrary $\mathbf{y}$, we can
set $y_1$ to the above value to yield the desired expression.

When $k=1$, $xy=0$ yields only one value for $y$.


\subsection{Problem 19}
The solution simply needs to be verified in this problem. We plug
$3\mathbf{c} = 4\mathbf{b}-\mathbf{a}$ and $3r=2|\mathbf{b}-\mathbf{a}|$ into
$|\mathbf{x}-\mathbf{c}|=r$.
\begin{IEEEeqnarray}{rCl}
    |\mathbf{x}-\mathbf{c}| & = & r \IEEEnonumber\\
    |\mathbf{x}-\frac{4}{3}\mathbf{b}+\frac{1}{3}\mathbf{a}| & = &
        \frac{2}{3}|\mathbf{b}-\mathbf{a}| \IEEEnonumber\\
    |\mathbf{x}|^2 - \frac{8}{3}\mathbf{xb} + \frac{2}{3}\mathbf{xa} +
        \frac{16}{9}|\mathbf{b}|^2 + \frac{1}{9}|\mathbf{a}|^2 +
        \frac{8}{9}\mathbf{ab} & = & \frac{4}{9}|\mathbf{b}|^2 -
        \frac{8}{9}|\mathbf{ab}| + \frac{4}{9}|\mathbf{a}|^2 \IEEEnonumber\\
    |\mathbf{x}|^2 - \frac{8}{3}\mathbf{xb} + \frac{2}{3}\mathbf{xa} +
        \frac{12}{9}|\mathbf{b}|^2 + \frac{1}{3}|\mathbf{a}|^2 & = & 0
        \IEEEnonumber\\
    3|\mathbf{x}|^2 - 8\mathbf{xb} + 2\mathbf{xa} + 4|\mathbf{b}|^2 +
        |\mathbf{a}|^2 & = & 0 \IEEEnonumber\\
    4|\mathbf{x}|^2 - 8\mathbf{xb}  4|\mathbf{b}|^2 + & = & |\mathbf{x}|^2 -
        2\mathbf{xa} + |\mathbf{a}|^2 \IEEEnonumber\\
    4(\mathbf{x}-\mathbf{b})^2 & = & (\mathbf{x}-\mathbf{a})^2\IEEEnonumber\\
    2|\mathbf{x}-\mathbf{b}| & = & |\mathbf{x}-\mathbf{a}| \IEEEnonumber
\end{IEEEeqnarray}


\subsection{Problem 20}
If we eliminate condition (III) from the definition of a cut, Step 2 and Step 3
are still valid, as they doesn't depend on (III). Thus we can still conclude
that $\mathbf{R}$ is an ordered set with the least upper bound property.

As for the addition properties of fields, the proofs for (A1) through (A3) from
the text still hold.

We redefine the zero element $0^*$ to be the set of all negative rational
numbers and the number $0$, which preserves properties (I) and (II) of being a
cut. Then the proof for (A4) in the text still holds once we account for the
number 0. For $r \in \alpha$, $r + 0 \leq r$, so $r + 0 \in \alpha$. And
because $s+0=s \ \forall s \in \alpha$, $s \in \alpha + 0$.

Trying to prove (A5) demonstrates why trying to define a $0^*$ with a largeset
element, 0, and thus, try to define cuts with property (III), fails to produce
a field. For if we defined $\gamma \in \mathbf{R}$ as the set of all rational
numbers, which is still a cut by our new relaxed definition, it is the case 
that there's no $\alpha$ such that $\alpha + \gamma = 0^*$, making (A5)
impossible to prove. $\gamma$ has no largest value, so $\alpha + \gamma$ has
no largest value, so it cannot be the case that $0^* \subset \alpha + \gamma$.


\end{document}
